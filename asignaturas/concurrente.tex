\section[An�lisis y dise�o de sistemas concurrentes]{Modelos para el an�lisis y dise�o de sistemas concurrentes}

\subsection{Descripci�n}

En esta asignatura se describen los principales modelos para la descripci�n de sistemas concurrentes, como son las �lgebras de procesos, redes de Petri y aut�matas de estados finitos. Tambi�n se abordan extensiones de estos modelos que incrementan su capacidad como son los modelos temporizados y con probabilidades.\\

Adicionalmente se presentan las principales herramientas que dan soporte a dichos modelos como por ejemplo \textit{Uppaal}\cite{uppaalweb} y cuales son las t�cnicas de an�lisis de propiedades que utilizan.\\

\subsection{Trabajo realizado}

En esta asignatura no han habido partes temporalmente diferenciadas, todos los contenidos se han dado simult�neamente lo cual permite comparar los distintos modelos entre s� y analizar sus ventajas e inconvenientes respecto a otros.\\

Entre los modelos que se presentan est�n las redes de Petri y las �lgebras de procesos, estas �ltimas pueden dar lugar a los modelos de aut�matas de estados finitos. Durante el curso el alumno realiza el modelado de varios casos de estudio, usando esos 3 modelos distintos y analizando las propiedades de los mismos.\\

Finalmente se realiza un ejercicio de modelado final utilizando las tres herramientas que se presentan en la asignatura (\textit{Tina}, \textit{Uppaal} y \textit{CWB}), que utilizan uno de los tres modelos presentados en el curso.

\subsection{Relaci�n con el tema de investigaci�n}
 
Lo visto en esta asignatura me sirve para modelar un sistema y que analizando su modelo pueda determinar sus propiedades m�s importantes. Estas propiedades pueden ser la aparici�n de bloqueos por ejemplo, o que un determinado elemento del programa llegue a ejecutarse alguna vez. Adem�s, mediante la validaci�n del modelo podemos determinar si realmente un programa se comporta como queremos.\\

Puedo por tanto modelar el comportamiento de cualquier software que dise�e durante la tesis. Esto me permitir�a evaluar su comportamiento teniendo en cuenta la mayor�a de casos posibles a los que puede llegar y que adem�s sea fiable.\\