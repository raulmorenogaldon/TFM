\section{Computaci�n en clusters}

\subsection{Descripci�n}

En esta asignatura se estudian y analizan los diferentes aspectos de un \textit{cluster} de computadores. Se describen las tendencias en cuanto a los nuevos sistemas de interconexi�n y I/O, como puede ser Infiniband. Adem�s se presentan las posibilidades y los problemas a resolver en cuanto a la configuraci�n de plataformas \textit{cluster}, mostrando ejemplos de aplicaciones que permiten aprovechar estas arquitecturas. Como �ltimo punto se introduce al alumno en la programaci�n paralela usando \textit{MPI}.\\

\subsection{Trabajo realizado}

En la primera parte de la asignatura se presenta al alumno un estado del arte sobre qu� son los \textit{cluster} y su impacto en el mercado, analizando la lista del \textit{Top 500} de computadores. Seguidamente se tratan los entornos software necesarios para que una arquitectura de tipo \textit{cluster} de la sensaci�n al usuario de estar trabajando con un �nico recurso computacional, lo cual se denomina imagen de sistema �nico.\\

En la parte final se presenta al alumno la librer�a \textit{MPI} y una peque�a introducci�n a su programaci�n orientada a \textit{cluster}. Para ello se le anima a programar un algoritmo basado en el recorrido de matrices usando esta librer�a y ejecut�ndolo con en un \textit{cluster} con distinto n�mero de nodos. Finalmente se hace uso de un software de monitorizaci�n llamado ``\textit{Paraver}'' analizando trazas de una aplicaci�n \textit{MPI} y permitiendo determinar cual ha sido el comportamiento durante su ejecuci�n.\\

Como trabajo final de asignatura, he realizado una introducci�n sobre la tecnolog�a \textit{SSE}\cite{ssemanual} (\textit{Streaming SIMD Extensions}) como uso del paralelismo a nivel de procesador en un \textit{cluster}, explicando en qu� consiste y peque�os ejemplos sobre su uso.\\

\subsection{Relaci�n con el tema de investigaci�n}
 
El uso de un \textit{cluster} es crucial para acelerar las aplicaciones de an�lisis de genoma, los grandes requisitos computacionales de estas requieren muchos recursos a su disposici�n, siendo un \textit{cluster} el que puede ofrecerlos. Por lo tanto se puede orientar el desarrollo del software al uso de estas plataformas que ofrecen un alto grado de paralelismo y adem�s distintos recursos, adecuados para distintas situaciones (procesadores, GPUs\dots).\\

Adem�s se puede utilizar \textit{MPI} para estos fines, ya que permite aprovechar los recursos disponibles y adem�s est� ampliamente documentado y soportado por la comunidad cient�fica.\\