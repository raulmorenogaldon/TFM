\section{Modelado y evaluaci�n de sistemas}

\subsection{Descripci�n}

Esta asignatura se centra en presentar al alumno una visi�n sobre el modelado de sistemas que le permita adem�s evaluar sus caracter�sticas, centr�ndose en el modelado para simulaci�n. El objetivo es el an�lisis de un sistema real, tanto est�tico como din�mico, para obtener un modelo que lo describa lo mejor posible.\\

Adem�s ese modelo es utilizable para simulaci�n, un proceso que lo utiliza para analizar y evaluar el rendimiento de un sistema antes de construirlo, evaluar las consecuencias de un suceso antes de que ocurra, comparar varios sistemas entre s�, conocer lo que ocurre realmente en el sistema\dots\\ 

\subsection{Trabajo realizado}

En la primera parte de la asignatura se presentan algunos m�todos para representar un sistema real en un modelo que sea analizable computacionalmente para determinar su comportamiento. Entre los modelos que se presentan se encuentran principalmente los que se utilizan para el campo de la simulaci�n, que permite determinar la evoluci�n de un sistema en el tiempo. Para ello hay que caracterizar el sistema que se quiere modelar, siendo de varios tipos: est�tico, din�mico, continuo, discreto\dots\\

En la segunda parte de la asignatura se presenta teor�a de colas como herramienta para evaluar un sistema del tipo servidor que utilice colas donde depositar los elementos a los que va a dar servicio. Mediante las m�tricas que ofrece se pueden evaluar tiempos de respuesta, dispersi�n de peticiones que pueden ser atendidas, ritmo m�ximo de llegada de esas peticiones para que puedan ser atendidas, etc.\\

En la ultima parte se inicia al alumno en el uso de dos simuladores de redes de interconexi�n: NS2\cite{ns2web} y OPNET\cite{opnetweb}. Estos dos son los simuladores m�s extendidos para este tipo de simulaci�n, permiti�ndote modelar una red tanto a nivel de topolog�a como a nivel hardware de conmutador.\\ 

Tambi�n se realiza un trabajo final que tenga que ver con los contenidos presentados en la asignatura, en mi caso el tema es el modelado del ADN para que sea analizable computacionalmente y una peque�a evaluaci�n de rendimiento del recalibrador que he dise�ado.\\

\subsection{Relaci�n con el tema de investigaci�n}

La relaci�n de esta asignatura con el an�lisis del ADN es directo, puesto que para analizarlo primero hay que modelarlo en la m�quina y a partir de ah� se podr�n comparar ADNs entre s�, simular las consecuencias de determinados cambios en la estructura del mismo, etc. Por otra parte tambi�n es �til para evaluar el rendimiento del sistema, cuya rapidez de respuesta debe ser lo m�s alta posible, evaluando cuellos de botella y c�digo que tiene que ser acelerado.\\