\section{Tecnolog�as de red de altas prestaciones}

\subsection{Descripci�n}

Esta asignatura pretende presentar el papel que las redes de interconexi�n tienen hoy d�a en la arquitectura de diversos sistemas distribuidos, desde los supercomputadores (miles de nodos de c�lculo unidos por una red de altas prestaciones), hasta los entornos Grid y Cloud, donde la interconexi�n es la propia Internet.\\

El objetivo de esta asignatura es la descripci�n de los aspectos m�s relevantes de una red de altas prestaciones. Tambi�n se analizan las alternativas de dise�o para los distintos elementos de las redes de interconexi�n, adem�s de comprender y distinguir los distintos tipos de arquitecturas de computaci�n distribuida de la actualidad.\\

\subsection{Trabajo realizado}

Durante la primera parte de esta asignatura se realizan diversos trabajos para obtener informaci�n sobre las redes que se utilizan en los mejores supercomputadores del mundo, obtenidos de las listas del Top 500. El objetivo es caracterizar estas redes y obtener sus principales propiedades, para determinar as� su adecuaci�n al entorno en que se estaban utilizando.\\

Otra parte de la asignatura consiste en la lectura, an�lisis, resumen y presentaci�n de diversos art�culos cient�ficos relacionados con los contenidos que se presentan en la misma. Esto permite adquirir pr�ctica en la lectura de art�culos, adem�s de obtener algunos conocimientos extra sobre redes de interconexi�n.\\

Para el trabajo final de asignatura he realizado una recopilaci�n de informaci�n sobre software de an�lisis de genoma que utilizaba sistemas distribuidos cloud para acelerar estos an�lisis. En el trabajo se describen detalles sobre c�mo utilizan los programas el cloud, y qu� elementos de �ste les daba la ventaja.\\

\subsection{Relaci�n con el tema de investigaci�n}

Para la computaci�n distribuida hay que utilizar de un modo u otro una red de interconexi�n que una todos los nodos de procesamiento, sea o no de altas prestaciones, por lo que los conceptos introducidos por esta asignatura para estas redes son directamente aplicables a mi investigaci�n. La red de interconexi�n es de los elementos m�s importantes a la hora de obtener beneficio en la computaci�n distribuida.\\

Adem�s, dependiendo del entorno de aplicaci�n, unas redes ser�n m�s adecuadas que otras, ya que no es lo mismo acelerar un programa para un \textit{cluster} con una red de interconexi�n determinada, que acelerarlo para una red \textit{on-chip}.\\

