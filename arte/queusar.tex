\chapter{Estado del arte: �Conclusiones?�Factores a tener en cuenta?�Lo meto en el capitulo de trabajo de investigaci�n?}

\section{Modelo de programaci�n paralela a tener en cuenta}

Es interesante la aplicaci�n de ambos modelos de programaci�n en la elaboraci�n de esta \textit{tesis}, a distintos niveles. Podemos utilizar el modelo de memoria compartida a nivel de procesador, puesto que hoy d�a los nuevos procesadores son mayoritariamente un sistema \textit{SMP} \textit{on-chip} con varios n�cleos de procesamiento y que comparten la misma memoria principal. Podemos usar este modelo de programaci�n para acelerar los tiempos de ejecuci�n dentro del procesador, usando con la m�xima eficiencia posible todos los n�cleos de procesamiento del mismo.\\

Podemos aplicar igualmente el modelo de paso de mensajes para llevar la ejecuci�n a un entorno multiprocesador o un \textit{cluster}. Mediante este modelo podemos acelerar el tiempo de ejecuci�n de los programas usando en paralelo los procesadores de los que se dispone en un sistema tipo \textit{cluster}. Con ello se busca mayor velocidad de computo cuantos m�s recursos (procesadores) disponga el sistema.\\

El modelo de \textit{PGAS} en cambio no se considera factible su uso en esta \textit{tesis}. Esto es debido a que la capa de virtualizaci�n, para dar la visi�n de memoria compartida, es costosa tanto econ�micamente como en prestaciones. Debido a la sobrecarga a la que somete al sistema y puesto que estamos en entornos de altas prestaciones, no se utilizar� al no considerarse necesaria.\\

Podemos concluir por tanto en el uso del modelo de memoria compartida combinado con el modelo de paso de mensajes, cada uno acelerando el tiempo de ejecuci�n a distintos niveles en un sistema multiprocesador.\\

\section{Tratamiento de datos}

\section{Software referencia de bioinform�tica para esta tesis}

En esta \textit{tesis} tenemos como referencia la \textit{suite} de herramientas del \textit{GATK} puesto que es la m�s utilizada. La raz�n de la elecci�n radica en que aun siendo la m�s utilizada, no es una herramienta pensada para ofrecer un alto rendimiento. A pesar de esto, sus ventajas son la facilidad a la hora de procesar cualquier \textit{dataset} de entrada, sea cual sea el secuenciador que lo haya generado. Esto unido a la capacidad para procesar el \textit{dataset}, sea del tama�o que sea, y la posibilidad de aplicarle cualquier algoritmo justifican el amplio uso de esta herramienta.\\

En cuanto a sus desventajas, la principal es la falta de rendimiento como ya se ha comentado. La herramienta en s� hace su trabajo y obtiene resultados, pero no es factible para una aplicaci�n cotidiana aplicada, por ejemplo, al �mbito de la sanidad (donde no es lo mismo que se obtengan los resultados de un paciente a los 2 meses que a las 2 horas).\\

Lo que se pretende por tanto es replicar esta herramienta para conservar sus ventajas pero intentando eliminar sus desventajas. Para ello se aplicar�n t�cnicas de computaci�n de altas prestaciones, usando lenguajes de programaci�n eficientes con memoria y los recursos del sistema y adem�s llevando la ejecuci�n al �mbito de los sistemas multiprocesador.\\