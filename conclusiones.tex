\chapter{Conclusiones}

En este cap�tulo se presentar�n las conclusiones a las que se ha llegado durante la realizaci�n de este trabajo.\\

\section{Conclusiones}

Actualmente se est� implementando SSE2 de forma manual en algunas partes de c�digo que son cr�ticas en el rendimiento, con la intenci�n de aprovechar al m�ximo los recursos de computo de los que se dispone. Esto, junto a la implementaci�n autom�tica de SSE y SSE2 y otras t�cnicas de optimizaci�n del compilador, permite aumentar la eficiencia como ya se ha visto en este cap�tulo. El siguiente paso puede ser utilizar SSE3 o SSE4, que no son implementadas autom�ticamente y pueden afectar al rendimiento. Otra raz�n para implementar manualmente SSE2, es que en procesadores de 32 bits, el compilador no genera c�digo SSE2 autom�ticamente, por lo que se perder�a esta ventaja. Tambi�n se perder�a si utilizamos otro compilador que tampoco genere este tipo de c�digo de forma autom�tica.\\

Durante la realizaci�n de este trabajo se ha logrado el objetivo de conseguir un recalibrador eficiente, aunque aun queda mucha funcionalidad por a�adirle. Este recalibrador ya explota estructuras de datos optimizadas y adem�s lo hace unas 4 veces m�s r�pido que GATK, por lo que se va por buen camino. Con la implementaci�n del modelo de memoria compartida con OpenMP y el de paso de mensajes MPI se espera obtener un gran aumento de rendimiento, adem�s de permitir la utilizaci�n de un supercomputador. Esto permitir�a realizar el descubrimiento de variantes en tiempo adecuado para que sea factible su aplicaci�n, por ejemplo, en el �mbito de la sanidad entre otras.\\