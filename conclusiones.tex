\chapter{Conclusiones generales}

En este cap�tulo se presentar� un resumen de las conclusiones a las que se ha llegado durante la realizaci�n de este trabajo.\\

Se han estudiado las principales tecnolog�as de computaci�n paralela y su programaci�n, ofreciendo una forma de acelerar el rendimiento de un software mediante su ejecuci�n en un entorno \textit{cluster}. Adem�s, podemos utilizar GPUs para acelerar a�n m�s los procesos que utilicen operaciones vectoriales masivas o que encajen bien con procesadores SIMD. Esto nos permitir� desarrollar herramientas eficientes que sean m�s r�pidas cuantos m�s procesadores tengan disponibles en un \textit{cluster}.\\

El tratamiento de datos cuando se manejan de forma masiva es cr�tico para el rendimiento. Un \textit{dataset} m�s grande que la memoria principal disponible supondr�a no poder cargarlo completamente en memoria y por tanto acceder continuamente a disco. Esto supone un decremento importante en las prestaciones. Por tanto se hace necesario un modo de tratar estos datos que permita obtener un acceso por partes, parecido al paradigma \textit{MapReduce}. Una de las soluciones, en un entorno \textit{cluster}, es usar las memorias de todos los nodos disponibles ya que se incrementa la probabilidad de que los datos se carguen en memoria completamente, y por tanto aumente el rendimiento.\\

Tambi�n se han visto herramientas utilizadas actualmente en bioinform�tica para el an�lisis de ADN. La mayor�a de ellas son secuenciales o permiten un m�nimo paralelismo de grano grueso utilizando hilos. Otras permiten su ejecuci�n en entornos \textit{cluster} pero son para aplicaciones muy espec�ficas, lo que obliga a ejecutarlas a modo tuber�a. Se ve por tanto una necesidad de tener un grupo de herramientas que permitan realizar an�lisis completos sin necesidad de herramientas extra. Adem�s estas herramientas ganar�an en prestaciones y permitir�an explotar el paralelismo en un sistema \textit{cluster}, permitiendo su uso cotidiano en diversos �mbitos de aplicaci�n.\\

La creaci�n de un conjunto de herramientas que cumplan los requisitos anteriores es factible. Para ello se utilizar�n tecnolog�as de paralelismo que exploten los recursos de c�mputo actuales. Se pretende crear este conjunto dentro de la comunidad cient�fica, ofreciendo a la misma la posibilidad de participar y realizar sus an�lisis sin restricciones.\\

Durante el primer a�o ya se han producido avances significativos. Se ha implementado una de las herramientas necesarias en el inicio del proceso de descubrimiento de variantes en C, utilizando estructuras de datos optimizadas para memoria. El rendimiento respecto a la herramienta que se ha tomado como referencia (GATK) es mucho mayor, incluso con esta utilizando hilos, lo cual nos indica el bajo grado de rendimiento que tienen actualmente estas herramientas.\\

Se han dado los primeros pasos y ya se han obtenido resultados espe\-ranzadores, si todo sale bien dentro de unos a�os podr�an utilizarse estas herramientas de forma cotidiana. Cada vez que un paciente fuese a la consulta del m�dico no tendr�a que preguntarle qu� le pasa, realizarle un an�lisis de su ADN ser�a suficiente para determinar qu� le ha pasado, qu� le pasa y qu� le va a pasar, del mismo modo que te hacen un an�lisis de sangre. Teniendo en cuenta adem�s que las aplicaciones de este tipo de herramientas no se limitan a la medicina, las posibilidades son muy amplias.\\

